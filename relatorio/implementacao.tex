\section{Implementa��o}

O trabalho consiste em identificar dinamicamente os par�metros do sistema para plantas de primeira, segunda e terceira ordem. Nesta se��o, iremos descrever a implementa��o em MATLAB dos algoritmos do gradiente normalizado e \textit{least-square} normalizado.

A primeira etapa da implementa��o � definir as condi��es iniciais, par�metros do sistema a serem identificados, fun��o de transfer�ncia da planta, filtro e sinal de entrada.

As condi��es iniciais do sistema s�o os valores para $y(0)$, $\theta_0$ e $P_0$, este �ltimo somente no caso do m�todo \textit{least-squares} normalizado. Consideremos uma planta est�vel de terceira ordem de forma a exemplificar com o sistema mais geral deste trabalho:
%
\begin{gather}
P(s) \, [y](t) = Z(s) \, [u](t) \, \\
\nonumber P(s) = (s+0.5)^3 = s^3 + 1.5 s^2 + 0.75s + 0.125 \, \\
\nonumber Z(s) = (s+0.3)^3 = s^2 + 0.6 s + 0.09 \, \\
\nonumber \overrightarrow{pz} =
\begin{bmatrix}
  z_2 & z_1 & z_0 & p_2 & p_1 & p_0
\end{bmatrix}
^\intercal=
\begin{bmatrix}
  1 & 0.6 & 0.09 & 1.5 & 0.75 & 0.125
\end{bmatrix}
^\intercal \, \\
\nonumber u(t) = r(t) = \mathrm{dc} + a_1 \, \mathrm{sin}(w_1 \, t) + a_2 \, \mathrm{sin}(w_2 \, t) + a_3 \, \mathrm{sin}(w_3 \, t) \,.
\end{gather}

O sinal de refer�ncia deve possuir pelo menos seis frequ�ncias para esse sistema, pois h� seis par�metros desconhecidos (Se��o 3.5 - Tao). Sendo assim, foi escolhido um sinal de refer�ncia com tr�s sen�ides de frequ�ncias distintas. Al�m disso, o sistema � dito com excita��o persistente se $u(t)$ n�o possui frequ�ncias do regressor $\phi(t)$. 
%
Agora, considere o seguinte filtro de terceira ordem:
%
\begin{gather}
\Lambda(s) = (s+1)^3 = s^3+3s^2+3s+1,\\
\nonumber \vec{\lambda} = 
\begin{bmatrix}
  \lambda_2 & \lambda_1 & \lambda_0
\end{bmatrix}
^\intercal=
\begin{bmatrix}
  3 & 3 & 1
\end{bmatrix}
^\intercal \,.
\end{gather}

O vetor de par�metros ideais � definido como:
%
\begin{gather}
\theta^* =
\begin{bmatrix}
  z_0 & z_1 & z_2 &(\lambda_0-p_0) & (\lambda_1-p_1) & (\lambda_{2}-p_{2})
\end{bmatrix}
^\intercal = \\
\nonumber \theta^* =
\begin{bmatrix}
  \textrm{flip}\left(\overrightarrow{pz}(1:3)\right)  & \textrm{flip}\left(\vec{\lambda}-\overrightarrow{pz}(4:6)\right)
\end{bmatrix}
^\intercal = \\
\nonumber \theta^* =
\begin{bmatrix}
  0.09 & 0.6 & 1 & 0.875 & 2.25 & 1.5
\end{bmatrix}
^\intercal \,.
\end{gather}

As equa��es do filtro $\frac{1}{\Lambda(s)}$ podem ser descritas como um sistema no espa�o de estados:
%
\begin{gather*}
\omega_1 = 
\begin{bmatrix}
  \omega_{11} & \omega_{12} & \omega_{13}
\end{bmatrix}
^\intercal \,, \\
%
\omega_2 = 
\begin{bmatrix}
  \omega_{21} & \omega_{22} & \omega_{23}
\end{bmatrix}
^\intercal \,,
\end{gather*}
%
\[
\begin{cases}
\dot{\omega}_{11} = \omega_{12}, \\
\dot{\omega}_{12} = \omega_{13}, \\
\dot{\omega}_{13} = u - \textrm{flip}(\vec{\lambda})^\intercal \, \omega_1
\end{cases}
\]

\[
\begin{cases}
\dot{\omega}_{21} = \omega_{22}, \\
\dot{\omega}_{22} = \omega_{23}, \\
\dot{\omega}_{23} = y - \textrm{flip}(\vec{\lambda})^\intercal \, \omega_2
\end{cases}
\]

Resumidamente:
%
\begin{gather}
\dot{\omega}_1 =
\begin{bmatrix}
  \omega_1(2:3) & y - \textrm{flip}(\vec{\lambda})^\intercal \omega_1
\end{bmatrix}
^\intercal,\\
%
\nonumber \dot{\omega}_2 =
\begin{bmatrix}
  \omega_2(2:3) & y - \textrm{flip}(\vec{\lambda})^\intercal \omega_2
\end{bmatrix}
^\intercal,\\
\phi = 
\begin{bmatrix}
  \omega_1 & \omega_2
\end{bmatrix}
\,.
\end{gather}

Podemos descrever a saída estimada e real do sistema pelos filtros:
%
\begin{gather}
y = \theta^{*\intercal} \, \phi \, \\
\hat{y} = \theta^\intercal \, \phi \, \\
\epsilon = \hat{y}-y = \tilde{\theta}^\intercal \, \phi \,.
\end{gather}

Finalmente, descrevemos a atualiza��o de $\theta$, os par�metros estimados,
conforme o gradiente normalizado ou pelo m�todo \textit{least-square}:
%
\[
\textrm{Gradiente Normalizado}
\begin{cases}
m^2 \leftarrow 1 + \phi^\intercal\phi\\
\dot{\theta} \leftarrow -\Gamma\ \phi \epsilon / m^2
\end{cases}
\]

\[
\textrm{Least-square Normalizado}
\begin{cases}
m^2 \leftarrow 1 + \phi^\intercal P\phi\\
\dot{\theta} \leftarrow -P \phi \epsilon / m^2\\
\dot{P} \leftarrow -P \phi \phi^\intercal P / m^2
\end{cases}
\]

Observe que no c�lculo de $m^2$, fixamos o valor do par�metro $\kappa$ em 1 para ambos os casos.
