%---------------------------------------------------------------------
\section{Resultados}

Avaliamos o funcionamento da fun��o que calcula os par�metros ideais de controle 2DOF plantas de ordens e graus relativos distintos, assim como diferentes polin�mios $A_0$ para o observador. 
%
Abaixo, fornecemos quatro exemplos para plantas de grau relativo $n^{*} = 1, 2, 3$ e $n^{*} = 4$. Para todas as plantas consideradas, obtivemos exatamente a mesma fun��o de transfer�ncia escolhida inicialmente como modelo de refer�ncia, o que mostra a corretude do algoritmo.

%---------------------------------------------------------------------
\subsection{Simula��o \#1 ($n = 2, n^{*} = 1, n_p = 4$)}

\begin{align*}
  P(s) &= \frac{2\left(s+1\right)}{s^2 -2s + 1}\,,  &  P_m(s) &= \frac{s+2}{s^2 + 4s + 1}\,, & A_0(s) &= 1
\end{align*}

Script de teste \textit{parameters\_214}:
%
\lstinputlisting[style=myMatlab]{../matlab/parameters_214.m}

A fun��o \textit{find2DOFparameters} retornou os seguintes valores:
%
\begin{align*}
 \theta_1^* = 1 \,, \qquad \theta_n^* = -3 \,, \qquad \theta_2^* = 6 \,, \qquad \theta_{2n}^* = 0.5 \qquad \Lambda(s) = s + 2
\end{align*}

%
A sa�da da fun��o \textit{calculate2DOFmodelTF} comprovou que esses valores encontrados resultam em $Y(s)/R(s) = P_m(s)$, igual ao modelo de refer�ncia proposto, dada a planta $P(s)$.
%

%---------------------------------------------------------------------
\subsection{Simula��o \#2 ($n = 3, n^{*} = 2, n_p = 5$)}

\begin{align*}
  P(s) &= \frac{2\left(s+1\right)}{s^3 +s^2 -2s + 1}\,,  &  P_m(s) &= \frac{1}{s^2 + 4s + 1}\,, & A_0(s) &= s+1
\end{align*}

Script de teste \textit{parameters\_325}:
%
\lstinputlisting[style=myMatlab]{../matlab/parameters_325.m}

A fun��o \textit{find2DOFparameters} retornou os seguintes valores:
%
\begin{align*}
 {\theta_1^*}^T = \mat{-4 & -4} \,, \qquad \theta_n^* = -3.5 \,, \qquad {\theta_2^*}^T = \mat{-0.5 & 5.5} \,, \qquad \theta_{2n}^* = 0.5 \qquad \Lambda(s) = s^2 + 2s + 1
\end{align*}

%
A sa�da da fun��o \textit{calculate2DOFmodelTF} comprovou que esses valores encontrados resultam em $Y(s)/R(s) = P_m(s)$, igual ao modelo de refer�ncia proposto, dada a planta $P(s)$.

%---------------------------------------------------------------------
\subsection{Simula��o \#3 ($n = 3, n^{*} = 3, n_p = 4$)}

\begin{align*}
  P(s) &= \frac{1}{s^3 + s^2 -2s + 1}\,,  &  P_m(s) &= \frac{1}{s^3 +2s^2 + 4s + 1}\,, & A_0(s) &= s^2 + 2s + 1
\end{align*}
%

Script de teste \textit{parameters\_334}:
%
\lstinputlisting[style=myMatlab]{../matlab/parameters_334.m}

A fun��o \textit{find2DOFparameters} retornou os seguintes valores:
%
\begin{align*}
 {\theta_1^*}^T = \mat{-1 & -7} \,, \qquad \theta_n^* = -4 \,, \qquad {\theta_2^*}^T = \mat{-1.5 & 7.5} \,, \qquad \theta_{2n}^* = 0.5 \qquad \Lambda(s) = s^2 + 2s + 1
\end{align*}

%
A sa�da da fun��o \textit{calculate2DOFmodelTF} comprovou que esses valores encontrados resultam em $Y(s)/R(s) = P_m(s)$, igual ao modelo de refer�ncia proposto, dada a planta $P(s)$.

%---------------------------------------------------------------------
\subsection{Simula��o \#4 ($n = 7, n^{*} = 4, n_p = 11$)}

\begin{align*}
  P(s) &= \frac{3\left(s^3 + 4s^2 + 2s + 4\right)}{s^7 + 3s^6 + s^5 - 2s^4 + 6s^2 -2s + 4}\,,  &  P_m(s) &= \frac{1.5 \left(s+2\right)}{s^5 + 5s^4 + 10.75s^3 +12.25s^2 + 7s + 1.5}\,, \\
  A_0(s) &= s^2 + 2s + 1
\end{align*}
%

Script de teste \textit{parameters\_7411}:
%
\lstinputlisting[style=myMatlab]{../matlab/parameters_7411.m}

A fun��o \textit{find2DOFparameters} retornou os seguintes valores:
%
\begin{align*}
 {\theta_1^*}^T &= \mat{-4 & -33.75 & -132.75 & -249.5 & -183.5 & -197} \,, & \theta_n^* &= -23.917 \,, \\
 {\theta_2^*}^T &= \mat{72.667 & 372.25 & 709.333 & 670.167 & 256.75 & 113.667} \,, & \theta_{2n}^* &= 0.5 \,, \\
 \Lambda(s) &= s^6 + 7s^5 + 20s^4 + 30s^3 + 25s^2 + 11s + 2
\end{align*}

%
A sa�da da fun��o \textit{calculate2DOFmodelTF} comprovou que esses valores encontrados resultam em $Y(s)/R(s) = P_m(s)$, igual ao modelo de refer�ncia proposto, dada a planta $P(s)$.
